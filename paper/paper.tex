\documentclass{article}

\usepackage{arxiv}

\usepackage[utf8]{inputenc} % allow utf-8 input
\usepackage[T1]{fontenc}    % use 8-bit T1 fonts
\usepackage{hyperref}       % hyperlinks
\usepackage{url}            % simple URL typesetting
\usepackage{booktabs}       % professional-quality tables
\usepackage{amsfonts}       % blackboard math symbols
\usepackage{amsmath}
\usepackage{nicefrac}       % compact symbols for 1/2, etc.
\usepackage{microtype}      % microtypography
\usepackage{graphicx}
\usepackage{natbib}
\usepackage{doi}
\usepackage{csquotes}


\def \thetitle {Correcting supernova luminosity for redshift implies no accelerated expansion}
\title{\thetitle}

\date{\today}

\author{
  \href{https://orcid.org/0000-0001-6450-3262}{\includegraphics[scale=0.06]{orcid.pdf}\hspace{1mm}Logan P.~Evans}
  \\ \texttt{loganpevans@gmail.com}
}

% Uncomment to override  the `A preprint' in the header
%\renewcommand{\headeright}{Technical Report}
%\renewcommand{\undertitle}{Technical Report}
\renewcommand{\shorttitle}{\textit{arXiv} Template}

%%% Add PDF metadata to help others organize their library
%%% Once the PDF is generated, you can check the metadata with
%%% $ pdfinfo template.pdf
\hypersetup{
pdftitle={\thetitle},
pdfsubject={astro-ph.CO},
pdfauthor={Logan P.~Evans},
pdfkeywords={cosmological parameters, dark energy},
}

\newtheorem{theorem}{Theorem}
\newtheorem{corollary}{Corollary}
\newtheorem{lemma}{Lemma}

\begin{document}
\maketitle

\begin{abstract}
  Existing analysis of Type Ia supernova relies on the assumption that
  luminosity is not affected by redshift. However, when luminosity is corrected
  for redshift, the relationship between luminosity distance and redshift for
  Type Ia supernova becomes linear. This implies that the expansion rate of the
  universe is not accelerating and there is no need for dark energy to explain
  observational data.
\end{abstract}

% keywords can be removed
\keywords{Cosmological Parameters \and Dark Energy \and Luminosity Distance}

\section{Introduction}

TODO: Discuss \citet{riess1998}, \citet{perlmutter1999}, and \citet{perlmutter2003}, as well as their nobel prize, summarizd in \citet{straumann2012}.

Summarize dark energy. Emphasize that dark energy is a popular explanation for
why distant supernova appear to be too far away.

Talk about the difficulty of curating supernova data, and summarize the work
done by \citet{betoule2014}.

\begin{displayquote}
\end{displayquote}

\section{Derivation of luminosity distance}

Luminosity distance, $D_L$, is the appearant distance of an object based on the
measured luminosity. This does not take into any movement of the observed
object between the time when the light was emitted and the light is observed.

Measurements of Type Ia supernova provide magnitude and redshift. To derive the
luminosity distance from these measurements, we start by computing the relative
brightness. Magnitude is defined on a logarithmic scale where magnitude 1 has
100 times the brightness of magnitude 6. To convert magnitude to relative brightness, we use:

\begin{equation}
  a
\end{equation}

\begin{equation}
\label{eq:g_def}
  G = \sqrt{ \bigg( s_i + \frac{s_j p_j}{p_i} \bigg)
             \bigg( s_j + \frac{s_i p_i}{p_j} \bigg)
           }
    = \frac{s_i p_i + s_j p_j}{\sqrt{p_i p_j}}.
\end{equation}

\section{Calibrated distance models}
\label{sec:disproof}

\section{Discussion of discrepancy}

\section{Conclusions}
What the heck?

\bibliographystyle{unsrtnat}
\bibliography{references}

\end{document}
